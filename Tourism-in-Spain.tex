% Options for packages loaded elsewhere
\PassOptionsToPackage{unicode}{hyperref}
\PassOptionsToPackage{hyphens}{url}
%
\documentclass[
]{article}
\usepackage{amsmath,amssymb}
\usepackage{iftex}
\ifPDFTeX
  \usepackage[T1]{fontenc}
  \usepackage[utf8]{inputenc}
  \usepackage{textcomp} % provide euro and other symbols
\else % if luatex or xetex
  \usepackage{unicode-math} % this also loads fontspec
  \defaultfontfeatures{Scale=MatchLowercase}
  \defaultfontfeatures[\rmfamily]{Ligatures=TeX,Scale=1}
\fi
\usepackage{lmodern}
\ifPDFTeX\else
  % xetex/luatex font selection
\fi
% Use upquote if available, for straight quotes in verbatim environments
\IfFileExists{upquote.sty}{\usepackage{upquote}}{}
\IfFileExists{microtype.sty}{% use microtype if available
  \usepackage[]{microtype}
  \UseMicrotypeSet[protrusion]{basicmath} % disable protrusion for tt fonts
}{}
\makeatletter
\@ifundefined{KOMAClassName}{% if non-KOMA class
  \IfFileExists{parskip.sty}{%
    \usepackage{parskip}
  }{% else
    \setlength{\parindent}{0pt}
    \setlength{\parskip}{6pt plus 2pt minus 1pt}}
}{% if KOMA class
  \KOMAoptions{parskip=half}}
\makeatother
\usepackage{xcolor}
\usepackage[margin=1in]{geometry}
\usepackage{longtable,booktabs,array}
\usepackage{calc} % for calculating minipage widths
% Correct order of tables after \paragraph or \subparagraph
\usepackage{etoolbox}
\makeatletter
\patchcmd\longtable{\par}{\if@noskipsec\mbox{}\fi\par}{}{}
\makeatother
% Allow footnotes in longtable head/foot
\IfFileExists{footnotehyper.sty}{\usepackage{footnotehyper}}{\usepackage{footnote}}
\makesavenoteenv{longtable}
\usepackage{graphicx}
\makeatletter
\def\maxwidth{\ifdim\Gin@nat@width>\linewidth\linewidth\else\Gin@nat@width\fi}
\def\maxheight{\ifdim\Gin@nat@height>\textheight\textheight\else\Gin@nat@height\fi}
\makeatother
% Scale images if necessary, so that they will not overflow the page
% margins by default, and it is still possible to overwrite the defaults
% using explicit options in \includegraphics[width, height, ...]{}
\setkeys{Gin}{width=\maxwidth,height=\maxheight,keepaspectratio}
% Set default figure placement to htbp
\makeatletter
\def\fps@figure{htbp}
\makeatother
\setlength{\emergencystretch}{3em} % prevent overfull lines
\providecommand{\tightlist}{%
  \setlength{\itemsep}{0pt}\setlength{\parskip}{0pt}}
\setcounter{secnumdepth}{5}
\ifLuaTeX
  \usepackage{selnolig}  % disable illegal ligatures
\fi
\IfFileExists{bookmark.sty}{\usepackage{bookmark}}{\usepackage{hyperref}}
\IfFileExists{xurl.sty}{\usepackage{xurl}}{} % add URL line breaks if available
\urlstyle{same}
\hypersetup{
  hidelinks,
  pdfcreator={LaTeX via pandoc}}

\author{}
\date{\vspace{-2.5em}}

\begin{document}

\begin{titlepage}
\centering
\vspace*{5cm}
{\Huge Tourism in Spain \par}
\vspace{2cm}
{\Large Nicolás Jiménez Muñoz and Aina Vila Arbusà \par}
\vfill
\end{titlepage}
\newpage

\tableofcontents

\newpage

\hypertarget{introduction}{%
\section{Introduction}\label{introduction}}

Spain is ranked as one of the world's top tourist destinations,
attracting millions of visitors every year due to its diverse
landscapes, rich gastronomy and cultural heritage. Tourism is a key
economic sector in Spain, significantly contributing to the country's
employment and. Understanding the fluctuations in tourist arrivals and
anticipating movements is very important for effective planning and
decision-making in both public and private sectors.

In this project, we will apply ARIMA models and the acquired knowledge
of time series analysis in order to generate meaningful and accurate
forecasts for future tourists arrivals. The main purpose is develop a
reliable forecasting model based on the available data of the selected
time series: the number of tourist arrivals in Spain (in millions).

By following Box-Jenkings methodology, which consists of a structured
process of description and model identification, estimation, diagnosis
and residual analysis, validation, final model selection and
forecasting, the project aims to provide both a comprehensive analysis
of the historical data and projections for future periods. This study
can be crucial for authorities and businesses to prepare for future
demand and optimizing their strategies accordingly.

\newpage

\hypertarget{basic-description-of-the-dataset}{%
\section{Basic description of the
dataset}\label{basic-description-of-the-dataset}}

The selected dataset contains monthly observations of the total number
of tourist arrivals in Spain, expressed in millions of visitors. It
spans from January 2000 to November 2018, comprising 227 observations.

The variable of interest is the number of tourist arrivals per month.
This provides valuable insights into seasonal patterns, trends and
fluctuations in the tourism sector, which are crucial for understanding
Spain's tourism industry.

The data is sourced from the Spanish Ministry of Industry, Commerce and
Tourism, specifically from the FRONTUR survey. The data collection
process, in this case, involves surveys and administrative records at
border entry points, ensuring comprehensive and accurate tracking of
international tourist movements. \newpage

\hypertarget{summary-of-the-methodology-used-for-data-analysis}{%
\section{Summary of the methodology used for data
analysis}\label{summary-of-the-methodology-used-for-data-analysis}}

In this analysis, the Box-Jenkins methodology is used to perform time
series forecasting employing ARIMA models, with additional extensions
for detecting and adjusting outliers and incorporating calendar effects.

ARIMA models (Autoregressive Integrated Moving Average) are statistical
tools that are used for analyzing and forecasting time series data.
These models take into account past values and prediction errors to make
predictions of future periods.

They consist of three key components:

\begin{itemize}
\item
  \textbf{AR (autoregressive)}: captures the linear relationship between
  the current value of the series and its past values.
\item
  \textbf{MA (moving average)}: models the relationship between the
  current value and past prediction errors.
\item
  \textbf{I (integrated)}: used to remove trends and achieve
  stationarity in the series.
\end{itemize}

The \textbf{Box-Jenkings} methodology involves several steps:

\begin{enumerate}
\def\labelenumi{\arabic{enumi}.}
\item
  \textbf{Identification}: to apply the ARIMA models, the time series
  must be stationary, meaning its statistical properties remain constant
  over time. First, a visual analysis is done to determine the
  appropriate transformations to achieve stationarity. Following this, a
  correlation analysis using Autocorrelation Function (ACF) and Partial
  Autocorrelation Function (PACF) is carried out to identify potential
  ARIMA models for the series.
\item
  \textbf{Estimation}: it involves estimating the parameters for the
  identified models. Then, a significant analysis of the parameters is
  conducted, where insignificant parameters are detected and,
  consequently, removed from the model.
\item
  \textbf{Validation}: it includes several aspects. One of them is the
  residual analysis to ensure they resemble white noise, meaning they
  are independent, have zero mean and constant variance. Additionally,
  model properties, invertibility and stability are assessed.
  Furthermore, so as to assess how well our model explains the data,
  metrics such as AIC, BIC, RMSE, MAE, RMSP and MAPE are used. Finally,
  the last 12 observations will be reserved to evaluate the model's
  forecasting accuracy. By comparing each model's performance, we will
  select the best one for real forecasting. In case the models are bad,
  the procedure has to be done again. Otherwise, the following step can
  be performed.
\item
  \textbf{Forecasting}: the chosen model is used to make predictions for
  the next 12 months beyond the last recorded observation.
\item
  \textbf{Calendar Effects}: to improve forecast accuracy,
  calendar-related factors such as public holidays or the number of
  weekends in a month are incorporated as external regressors. These
  effects are modeled, tested for significance, and used to adjust the
  series accordingly.
\item
  \textbf{Outlier Treatment}: after calendar adjustment, the outdetec
  function is applied to detect outliers in the data, which can be
  classified in three types:

  \begin{itemize}
  \item
    AO (additive outlier): effects a single observation.
  \item
    TC (temporary change): has a transitory effect that disappears over
    time.
  \item
    LS (level shift): causes a permanent change in the level of the
    series.
  \end{itemize}
\end{enumerate}

By carefully analyzing these outliers, their impact on the model is
assessed. The time series is then adjusted accordingly, and the
linearized time series is obtained. The forecasting process is repeated
using the transformed model, and the results of the new forecasts are
compared with those from the initial one to evaluate improvements.

\newpage

\hypertarget{results-and-interpretation}{%
\section{Results and interpretation}\label{results-and-interpretation}}

\hypertarget{identification}{%
\subsection{Identification}\label{identification}}

\hypertarget{transformations-for-stationarity}{%
\subsubsection{Transformations for
stationarity}\label{transformations-for-stationarity}}

Plotting the time of tourist arrivals in Spain makes easier to detect an
increasing trend, particularly evident since 2013. This upward
trajectory reflects the growth in tourism over the years.

Additionally, a clear seasonal pattern is observed, with peaks each year
that can be attributed to the increase in tourism during summer holiday
period. This seasonal variation highlights the cycle nature of tourism,
with higher numbers of arrivals during warmer months.

\begin{center}\includegraphics[width=1\linewidth]{Tourism-in-Spain_files/figure-latex/unnamed-chunk-1-1} \end{center}

In time series analysis, as it has been explained, it is essential for
the series to be stationary, meaning its statistical properties do not
change over time.

The first step in preparing a time series for modeling is to transform
it into a stationary form. To achieve this, we address three different
aspects.

\begin{enumerate}
\def\labelenumi{\arabic{enumi}.}
\tightlist
\item
  \textbf{Constant variance}: although the initial time series plot
  already suggests that variance might not be constant over time,
  building a box plot by periods offers a clearer visualization of how
  variance behaves throughout years. It can be seen in the first
  picture.
\end{enumerate}

The height of the boxes, which represents the interquartile range (IQR),
tends to increase in the later years, indicating growing dispersion in
the data. However, there are some exceptions, such as 2009 and 2004,
where variance appears unusually low. In 2009, the lower variance can
likely be attributed to the global financial crisis that started in
2008. During economic downturns, travel tends activity to decrease
sharply worldwide due to reduced income and economic uncertainty. As a
result, not only did tourism numbers decline, but they also fluctuated
less, as the impact was global and affected the general population.
Similarly, in 2004, the Madrid train bombings in March led to a
temporary shock in tourism due to concerns over safety and security.

Nevertheless, excluding these exceptional events, the overall pattern
suggests a potential heteroscedasticity in the series, meaning that the
variance changes over time. To stabilize the variance, a logarithmic
transformation is applied to the series.

We repeat the plots and, as a result, we see that the issue has been
fixed. The changes can be seen in the two plots below.

\begin{center}\includegraphics[width=1\linewidth]{Tourism-in-Spain_files/figure-latex/unnamed-chunk-2-1} \end{center}

\begin{center}\includegraphics[width=1\linewidth]{Tourism-in-Spain_files/figure-latex/unnamed-chunk-3-1} \end{center}

\begin{enumerate}
\def\labelenumi{\arabic{enumi}.}
\setcounter{enumi}{1}
\tightlist
\item
  \textbf{Seasonal pattern}: as previously mentioned, there is a clear
  seasonal pattern in the data. However, to better highlight the
  pattern, we can revise the monthplot of the series (left plot)
\end{enumerate}

Each year, peaks are observed during summer months, especially in July
and August, likely due to the increase in tourism during the holiday
period.

Consequently, we have to apply a seasonal difference to the series and
we repeat the monthplot (figure in the middle) and the plot of the
transformed series (figure on the right side), where we can see that the
transformation has effectively attenuated the seasonal fluctuations in
the series.

\begin{center}\includegraphics[width=1\linewidth]{Tourism-in-Spain_files/figure-latex/unnamed-chunk-4-1} \end{center}

\begin{enumerate}
\def\labelenumi{\arabic{enumi}.}
\setcounter{enumi}{1}
\tightlist
\item
  \textbf{Constant mean}: the final aspect to be checked is the presence
  of a trend. As it has been mentioned at the beginning, it seems, at
  first glance, that there the real series shows an upward trajectory
  that reflects the growth in tourism over the years.
\end{enumerate}

In order the series to be stationary, it is necessary to remove this
trend component. However, this should be done carefully, ensuring that
variance doesn't increase as a result of the transformation. We proceed
to visually inspect the transformed data obtained in the previous steps
(with logarithmic transformation and a seasonal difference).

Since determining the need for a regular difference is not
straightforward, we evaluate the variance of the series under the
different transformations: logarithmic transformation, logarithmic
transformation + seasonal difference and logarithmic transformation +
seasonal + regular difference.

Comparing the variances, we observe that in the previous step the
seasonal difference has decreased variance, while adding a regular
difference results in an increase in variance, suggesting that it may
not be beneficial.

\begin{longtable}[]{@{}lr@{}}
\toprule\noalign{}
Transformation & Variance \\
\midrule\noalign{}
\endhead
\bottomrule\noalign{}
\endlastfoot
lnseries & 0.156137 \\
d12lnseries & 0.004071 \\
diff(d12lnseries) & 0.004267 \\
\end{longtable}

\textbf{Series:} Based on the analysis, we conclude that the optimal
approach is to use the series with both logarithmic transformation and
seasonal difference. This version of the series will serve as our
stationary series for modeling.

The expression for the stationary series (\(W_t\)) in function of the
original series (\(X_t\)) and the backshift operator (\(B\)), is written
as:\[W_t = (1-B^{12})log(X_t)\]

\hypertarget{acf-and-pacf-analysis}{%
\subsubsection{ACF and PACF analysis}\label{acf-and-pacf-analysis}}

After applying the logarithmic transformation and one seasonal
difference, we are working with a model of the form:
\[ARIMA(p, 0, q)(P, 1, Q)_{12}\].

To determine the appropriate model parameters, we examine the
autocorrelation function (ACF) and the partial autocorrelation function
(PACF) plots shown above.

For the \textbf{seasonal part} (represented by the uppercase letters P
and Q), we focus on the lags colored in red, corresponding to multiples
of the seasonal period (lag 12, 24\ldots).

It is difficult to detect a clear decreasing pattern. We observe that
the seasonal lags in the PACF plot cut off after lag 2, suggesting a
seasonal autoregressive structure. We interpret this pattern as P = 2,
meaning a seasonal AR(2) component. Nevertheless, since the second
significant lag is very close to the blue confidence line, we can also
propose an AR(1), P = 1.

Alternatively, we can consider the PACF to have the infinite lags
different from zero. We propose a MA(0) due to the lack of significant
lags multiple to the seasonal part in the ACF plot.

On the other hand, for the \textbf{regular part}, at first glance the
autocorrelation function (ACF) shows that for the first few lags (up to
around lag 12), the autocorrelations are strong and significant.
Nevertheless, it shows a slowly decreasing pattern, gradually tapering
off which is typical of an AR process. Since, the PACF has the last
significant lag at position 3, p = 3 and, consequently we propose a
regular AR(3) component.

Based on the above, we have several model proposals:

\begin{itemize}
\tightlist
\item
  \(ARIMA: (3, 0, 0)(2, 1, 0)_{12}\)
\item
  \(ARIMA: (3, 0, 0)(1, 1, 0)_{12}\)
\item
  \(ARIMA: (3, 0, 0)(0, 1, 0)_{12}\)
\end{itemize}

Nevertheless, as we have mentioned, the ACF shows a suspicious shape
with too many significant lags, which leads us to thing that it might be
either because of non-stationarity of the transformed series or outliers
that could be affecting the autocorrelation structure.

\begin{center}\includegraphics[width=1\linewidth]{Tourism-in-Spain_files/figure-latex/unnamed-chunk-6-1} \end{center}

In order to make sure that we are doing a correct analysis, we can use
an alternative approach: the \textbf{auto.arima()} function from the
forecast package in R.

This function automates the process of model selection by iterating over
different combinations of parameters and selecting the best ARIMA model
based on criteria such as the AIC. It has been proven to be a more
efficient way of identifying an appropriate model, especially in cases
where manually analyzing ACF and PACF plots might be ambiguous.

See Figure 1.

The automatic function, suggests an \(ARIMA(2, 1, 0)(2, 0, 0)_{12}\).
The non-seasonal component of the model has AR(2) with a regular
differencing of 1 (d = 1), and the seasonal component suggests AR(2)
without seasonal differencing (since we have already applied a seasonal
difference of 12 in the data).

Therefore, taking into consideration the already done transformations we
have an \(ARIMA(2, 1, 0)(2, 1, 0)_{12}\) proposal.

Going back to the analysis we have done when regular differencing is
applied, the increase in variance from this operation might be
negligible and, indeed, an appropriate transformation to be carried out.

As a result, we change our working series and we now work with a series
that has gone through a logarithmic transformation, a seasonal
differencing and a regular differencing.

\begin{center}\includegraphics[width=1\linewidth]{Tourism-in-Spain_files/figure-latex/unnamed-chunk-7-1} \end{center}

We repeat the ACF and PACF analysis. We are now working with a model of
the form: \[ARIMA(p, 1, q)(P, 1, Q)_{12}\].

At first glance, we notice that ACF no longer displays the strong
autocorrelations in the first few lags that were previously present,
allowing for a more reliable analysis.

For the \textbf{seasonal part}, we focus on the lags highlighted in red.

It is still challenging to detect a clear, decreasing pattern again.
However, in the PACF plot, the seasonal lags appear to cut off after lag
1, suggesting a seasonal autoregressive structure. We interpret this as
P = 1, indicating a seasonal AR(1). Nevertheless, after the first two
lags (12 and 24), there are still some observations out of the
significance bounds. Although these are not very prominent, it has to be
taken into consideration.

This observation, leads us to an alternative proposal. If we view the
PACF as displaying many significant lags, while the ACF shows the last
significant seasonal lag at lag 1, this could support the idea of a
MA(1). However, it is difficult to make a consistent diagnosis for this
part.

Turning on the \textbf{regular part}, now we can observe a decreasing
pattern in the PACF, especially within the first year of observations.
So, if we focus on the last lag different from zero in the ACF function
it corresponds to lag 4. Since lag 2 is not significant but lags 3 and 4
are, we propose a MA(4) for this part.

However, if we revise again the plots we can also deduce a decreasing
tendency in the ACF, while the PACF shows that the first significant lag
is at lag 2, suggesting an AR(2) structure.

Based on what we have said, we now have new model proposals:

\begin{itemize}
\tightlist
\item
  \(ARIMA: (2, 1, 0)(2, 1, 0)_{12}\) \{proposed by the automatic
  function\}
\item
  \(ARIMA: (0, 1, 4)(1, 1, 0)_{12}\)
\item
  \(ARIMA: (2, 1, 0)(1, 1, 0)_{12}\)
\item
  \(ARIMA: (0, 1, 4)(0, 1, 1)_{12}\)
\item
  \(ARIMA: (2, 1, 0)(0, 1, 1)_{12}\)
\end{itemize}

\begin{center}\includegraphics[width=1\linewidth]{Tourism-in-Spain_files/figure-latex/unnamed-chunk-8-1} \end{center}

\hypertarget{estimation}{%
\subsection{Estimation}\label{estimation}}

We now proceed with the estimation of the models. Since our series
d1d12lnseries has already been differenced, with both seasonal and
regular differencing applied beforehand, we set d = 0 and D = 0 in the
model specification for the estimation and we use the transformed
stationary series to obtain the mean estimation (Fig 2)

In order to determine if the mean (intercept) is significant in the
proposed models we do a t-test:

\begin{itemize}
\item
  Hypothesis: \[H_0 : \mu_{W_t} = 0\] \[H_1 : \mu_{W_t} \neq 0\]
\item
  Test Statistic:
  \[\hat{t} = \frac{\hat{\mu} - 0}{S_{\hat{\mu}}} \sim N(0,1)\]
\item
  Decision rule: \[|\hat{t}| > 2 \Rightarrow H_1\]
  \[|\hat{t}| \leq 2 \Rightarrow H_0\]
\end{itemize}

Since means are not significant (Fig 3), we re-estimate the model for
the original series (ln transform) specifying the differentiations in
the estimation method (Fig 4) and we verify the significance of
coefficients (Fig 5, 6, 7, 8, 9).

The coefficients of models 1, 3 and 5, which are
\(ARIMA: (2, 1, 0)(2, 1, 0)_{12}\) \{proposed by the automatic
function\}, \(ARIMA: (2, 1, 0)(1, 1, 0)_{12}\) and
\(ARIMA: (2, 1, 0)(0, 1, 1)_{12}\), respectively, are all significant.

On the other hand, model 2 (\(ARIMA: (0, 1, 4)(1, 1, 0)_{12}\)) has
several non-significant coefficients: MA(2), MA(3) and MA(4). The least
significant is MA(3), with a t-value = 0.8009. In order to fix it, we
remove this coefficient and re-estimate the model.

Similarly, model 4 (\(ARIMA: (0, 1, 4)(0, 1, 1)_{12}\)) has
non-significant coefficients, MA(2) and MA(3). In this case, MA(2) is
the least significant, and it will be removed in the re-estimation.

After the re-estimation (Fig 10), t-test is done again.

Removing the non-significant coefficients mentioned is not sufficient
for any of these models, as they still include some that are not
statistically significant: specifically, MA(4) coefficient in model 2
(Fig 11), and MA(3) and MA(4) coefficients in model 3 (Fig 12).

To improve the models, we are will reduce the order of model 2 by
decreasing its MA degree to eliminate the non-significant MA(4) term.
For model 3, we will remove MA(3) term, since it is the least
significant among the remaining coefficients.

After the new re-estimation (Fig 13), t-test is carried out again (Fig
14, 15).

Since new non-significant coefficients arise in both models after
removing some of the initial ones, this indicates potential
overparametrization. Removing parameters can cause the model to
redistribute variance explanation, which leads to other parameters
becoming non-significant. This suggests that the model specification may
not be appropriate.

For this reason, these two models are ruled out, and we will focus on:

\begin{itemize}
\item
  Models 1: \textbf{\(ARIMA: (2, 1, 0)(2, 1, 0)_{12}\)}
\item
  Model 3: \textbf{\(ARIMA: (2, 1, 0)(1, 1, 0)_{12}\)}
\item
  Model 5: \textbf{\(ARIMA: (2, 1, 0)(0, 1, 1)_{12}\)}
\end{itemize}

\hypertarget{validation}{%
\subsection{Validation}\label{validation}}

\hypertarget{residual-analysis}{%
\subsubsection{Residual analysis}\label{residual-analysis}}

In order to do the validation of the models we first need to check if
the residuals behave like white noise:

\begin{itemize}
\item
  \textbf{Homocedasticity (constant variance)}
\item
  \textbf{Normality}
\item
  \textbf{Independence}
\end{itemize}

In order to verify if \textbf{variance} is constant, we start by
plotting the residuals.

In all three models, the residuals behave very similarly. Most of the
residuals lie within the confidence bounds, except for one observation
between 2002 and 2003, which appears to deviate from the overall
pattern, this can be considered to be an outlier that we are going to
take into account from now on.

Among the models, model 1 shows slightly narrower residual dispersion,
indicating potentially better variance behavior.

\begin{center}\includegraphics[width=1\linewidth]{Tourism-in-Spain_files/figure-latex/unnamed-chunk-10-1} \end{center}

To further examine variability, we plot the square root of the absolute
residuals and add a smooth trend line to observe potential changes in
the spread of residuals over time.

In the ideal case of homoscedastic variance, we would expect to see a
flat and stable smooth line.

Nevertheless, we observe a slight increase in variance during the
initial years of the data, between 2000 and 2005, where we have
previously observed the potential outlier. After this period, the line
flattens and stabilizes in all three models, achieving a more stable
variance in the later observations.

We can consider that variance is constant.

\begin{center}\includegraphics[width=1\linewidth]{Tourism-in-Spain_files/figure-latex/unnamed-chunk-11-1} \end{center}

Then, to assess the \textbf{normality} of residuals, we proceed with the
Q-Q plot analysis. On the left there is the mod1, in the middle mod3 and
finally mod5.

Across all three models, the residual patterns are fairly consistent.
The tails show deviations from normality, showing volatility, with model
3 displaying higher degree of dispersion compared to others.

Particularly there is the presence of a single, isolated observation
that lies far from the reference line, strongly indicating the influence
of an outlier.

\begin{center}\includegraphics[width=1\linewidth]{Tourism-in-Spain_files/figure-latex/unnamed-chunk-12-1} \end{center}

To further examine the normality of the residuals, we also plot the
histogram.

In this case, we can see a small, isolated bar on the left tail of the
histogram, which is likely associated with the isolated dot identified
in the Q-Q plot. Additionally, it has to be mentioned that the central
bars around the mean appear more concentrated and taller than the
reference normal curve, suggesting a distribution that is slightly more
peaked than the normal one.

These observations suggest that the presence of outliers could be
altering the residual distribution. Probably, addressing them we could
improve normality and overall model adequacy.

\begin{center}\includegraphics[width=1\linewidth]{Tourism-in-Spain_files/figure-latex/unnamed-chunk-13-1} \end{center}

The previous graphical observations can make us think that the normality
assumption might not hold for any of these three models, primarily due
to the presence of outliers. To formally and numerically test it, we
conduct the Shapiro-Wilk test on the residuals (Fig 16). In all three
cases, p-value is below the 0.05 significance threshold, indicating
evidence against normality assumption.

However, it is important to recognize that the Shapiro-Wilk test is
sensitive to outliers. A single outlier can alter the test outcome,
leading to a false rejection of normality. Considering what we have
observed on the previous plot of residuals, where we identified certain
isolated points that seem to behave differently from the rest, we can
say that probably addressing these outliers could result in residuals
that follow a normal distribution.

Finally, to assess if the model adequately captures the autocorrelation
structure of data, we examine the \textbf{independence} of residuals.

First, we begin by plotting the ACF and PACF of the residuals. Ideally,
if the model captures time dependence correctly, all autocorrelations
should lie within the confidence bands.

In our case, the three models look pretty similar. We observe a few
significant spikes at higher lags, far from the present. These, can be
considered satellites, that suggest that the model may not fully explain
the temporal structure of the data (Fig 17, 18, 19).

Moreover, to complement our assessment of residual independence, we
conduct a Ljung-Box test (Fig 20, 21, 22)

This test evaluates whether any group of autocorrelations of the
residuals is significantly different from zero. In all three models,
from lag 48 onwards, the p-values remain consistently below the
reference line, suggesting that there is a lack of independence among
the residuals. This, again, may be due to the presence of an outlier
that could be affecting the residuals' autocorrelation structure.

For earlier lags, most p-values are above 0.05, indicating that the
models adequately capture the autocorrelation structure of the
residuals. However, in models 1 and 3, a few observations are very close
to the threshold.

Nevertheless, given that outiler treatment could fix the mentioned
issues, we can assume that residuals are independent in all three models
once the outliers are properly handled.

After conducting the residual analysis, we can confidently conclude
that, after an appropriate outlier treatment, the residuals will likely
satisfy the assumptions of constant variance, normality and indepence,
behaving like white noise.

\hypertarget{model-expressions}{%
\subsubsection{Model expressions}\label{model-expressions}}

To express the structure of each model formally, we use the backshift
operator B. The general form of a seasonal ARIMA model is: \[
\phi_{p}(B)\, \Phi_{P}(B^s)(1 - B)^d(1 - B^s)^D X_t = \theta_{q}(B)\, \Theta_{Q}(B^s)\, Z_t
\] Now we are going to analise our three models one by one. :

\begin{itemize}
\tightlist
\item
  \textbf{Model 1}: \(ARIMA(2, 1, 0)(2, 1, 0)_{12}\)
  \[(1 - \phi_{1}B - \phi_{2}B^2)(1 - \Phi_{1}B^{12} - \Phi_{2}B^{24})(1 - B)(1 - B^{12})X_t = Z_t\]
  This model includes both non-seasonal and seasonal autoregressive
  terms of order 2, with first-order regular and seasonal differencing.
  There are no moving average (MA) components. Because of this,
  invertibility is not a concern in this case. However, the model admits
  an infinite moving average representation.
\end{itemize}

In order to assess stationarity and stability of the model, using the
characteristic polynomials of the AR and MA components, we are going to
calculate the modulus of their roots.

For the model to be stationary, all roots must lie outside the unit
circle. Since the condition is met, we can conclude that model 1 is
well-specified (Fig 23).

In the following plot we can visualize it. Since we are doing the
\textbf{inverse}, all roots are inside the unit circle as expected.

\begin{center}\includegraphics[width=1\linewidth]{Tourism-in-Spain_files/figure-latex/unnamed-chunk-14-1} \end{center}

\begin{itemize}
\tightlist
\item
  \textbf{Model 3}: \(ARIMA: (2, 1, 0)(1, 1, 0)_{12}\)
  \[(1 - \phi_{1}B - \phi_{2}B^2)(1 - \Phi_{1}B^{12})(1 - B)(1 - B^{12})X_t = Z_t\]
  This one is a simpler seasonal structure with a seasonal AR term at
  lag 12. Like the previous one, it admits an infinite MA
  representation.
\end{itemize}

In order to assess stationarity of the model, we examine the roots of
the characteristic polynomial. Since invertibility is not applicable, we
are going to focus on the autoregressive operators.

Since all roots lie outside the unit circle (Fig 24), we can conclude
that model 3 is stable and well-specified.

\begin{center}\includegraphics[width=1\linewidth]{Tourism-in-Spain_files/figure-latex/unnamed-chunk-15-1} \end{center}

\begin{itemize}
\tightlist
\item
  \textbf{Model 5}: \(ARIMA: (2, 1, 0)(0, 1, 1)_{12}\)
  \[(1 - \phi_{1}B - \phi_{2}B^2)(1 - B)(1 - B^{12})X_t = (1 + \Theta_{1}B^{12}) Z_t\]
  This model differs from the previous ones as it contains a seasonal
  moving average component (MA) instead of a seasonal AR term.
\end{itemize}

Because of the presence of both non-seasonal AR terms and seasonal MA
term, we must check stationarity and invertibility conditions.

To ensure causality and stationarity, the roots of the non-seasonal AR
polynomial must lie outside the unit circle as well as the roots ot the
seasonal MA polynomial, to ensure invertibility.

In this case, all roots have modulus greater than 1 (Fig 25).
Consequently, we can conclude that the AR part is causal, the MA
invertible and the model well-defined and stable.

\begin{center}\includegraphics[width=1\linewidth]{Tourism-in-Spain_files/figure-latex/unnamed-chunk-16-1} \end{center}

Then, we assess and compare the model fit metrics for the three models.

Examining the AIC and BIC values, we observe that all three models have
very similar metrics. Nevertheless, Model 5, has the lowest AIC and BIC
values, indicating a better fit relative to other models. These results
suggest that model 5 strikes a favorable balance between goodness of fit
and model complexity, making it a more efficient choice for this data
set.

\begin{longtable}[]{@{}lrr@{}}
\toprule\noalign{}
Model & AIC & BIC \\
\midrule\noalign{}
\endhead
\bottomrule\noalign{}
\endlastfoot
Model 1 & -692.7635 & -675.9336 \\
Model 3 & -686.9478 & -673.4839 \\
Model 5 & -692.7936 & -679.3297 \\
\end{longtable}

\hypertarget{forecasting-accuracy}{%
\subsubsection{Forecasting accuracy}\label{forecasting-accuracy}}

In order to assess model stability and forecasting performance, we
reserve the last 12 observations for validation. These are excluded from
model training and used to evaluate how well each model predicts unseen
data.

Down below, we present plots comparing the actual series with the
forecasts produced by each model. The black line represents the original
time series, the red line shows the forecasted values for the last 12
periods, and the blue lines indicate the 95\% confidence intervals.
Predictions from all models closely follow the same pattern, with some
deviations, specially in predictions for summer months, where the peak
is observed.

At the end, a summary table shows the main evaluation metrics: RMSE,
MAE, RMSPE, MAPE and AIC for both the full and the cut models, residual
variance, and the mean width of the confidence intervals.

As far as model stability is concerned, we observe that the differences
between the AIC values of the full and cut models are consistent for the
three models. This suggests that the model structures remain stable even
when we have less data. Residual variances (sig2\_f and sig2\_c) also
support this idea, since they remain almost the same across full and cut
versions for each model.

When it comes to forecasting accuracy, all three models have almost
identical RMSE, MAE, RMSPE and MAPE values. Although differences are
minimal, model 5 has the lowest error metrics. Since percentages are
lower than 10\%, our models are making reasonably accurate forecasts.

On the other hand, the mean width of the confidence intervals are also
very similar but, in this case, model 1 produces slightly narrower
intervals compared to others.

All the three models exhibit acceptable performance metrics.
Nevertheless, since we need to select a single model for forecasting, it
is essential to consider all the evaluation criteria discussed
previously.

First, we can rule out model 3, as it is the one with weakest
performance across the metrics.

If we compare model 1 and 5, model 5 has the lowest AIC for the full
model and the lowest error metrics. However, model 1 slightly
outperforms model 5 in terms of AIC for the cut model, as well as sigma
squared and the width of the confidence intervals. Given that the
differences where model 5 is better are relatively small compared to
those where model 1 performs better, the choice finally falls on model
1: \(ARIMA(2, 1, 0)(2, 1, 0)_{12}\)

\begin{center}\includegraphics[width=1\linewidth]{Tourism-in-Spain_files/figure-latex/unnamed-chunk-22-1} \end{center}

\begin{longtable}[]{@{}
  >{\raggedright\arraybackslash}p{(\columnwidth - 18\tabcolsep) * \real{0.1026}}
  >{\raggedleft\arraybackslash}p{(\columnwidth - 18\tabcolsep) * \real{0.1282}}
  >{\raggedleft\arraybackslash}p{(\columnwidth - 18\tabcolsep) * \real{0.1282}}
  >{\raggedleft\arraybackslash}p{(\columnwidth - 18\tabcolsep) * \real{0.0897}}
  >{\raggedleft\arraybackslash}p{(\columnwidth - 18\tabcolsep) * \real{0.0897}}
  >{\raggedleft\arraybackslash}p{(\columnwidth - 18\tabcolsep) * \real{0.0897}}
  >{\raggedleft\arraybackslash}p{(\columnwidth - 18\tabcolsep) * \real{0.0897}}
  >{\raggedleft\arraybackslash}p{(\columnwidth - 18\tabcolsep) * \real{0.0897}}
  >{\raggedleft\arraybackslash}p{(\columnwidth - 18\tabcolsep) * \real{0.0897}}
  >{\raggedleft\arraybackslash}p{(\columnwidth - 18\tabcolsep) * \real{0.1026}}@{}}
\toprule\noalign{}
\begin{minipage}[b]{\linewidth}\raggedright
Model
\end{minipage} & \begin{minipage}[b]{\linewidth}\raggedleft
AIC\_full
\end{minipage} & \begin{minipage}[b]{\linewidth}\raggedleft
AIC\_cut
\end{minipage} & \begin{minipage}[b]{\linewidth}\raggedleft
sig2\_f
\end{minipage} & \begin{minipage}[b]{\linewidth}\raggedleft
sig2\_c
\end{minipage} & \begin{minipage}[b]{\linewidth}\raggedleft
RMSE
\end{minipage} & \begin{minipage}[b]{\linewidth}\raggedleft
MAE
\end{minipage} & \begin{minipage}[b]{\linewidth}\raggedleft
RMSPE
\end{minipage} & \begin{minipage}[b]{\linewidth}\raggedleft
MAPE
\end{minipage} & \begin{minipage}[b]{\linewidth}\raggedleft
CIWidth
\end{minipage} \\
\midrule\noalign{}
\endhead
\bottomrule\noalign{}
\endlastfoot
Model 1 & -692.7635 & -651.8284 & 0.0022 & 0.0022 & 0.5535 & 0.4098 &
0.0602 & 0.0497 & 2.0162 \\
Model 3 & -686.9478 & -645.2995 & 0.0023 & 0.0023 & 0.5524 & 0.4193 &
0.0612 & 0.0511 & 2.0388 \\
Model 5 & -692.7936 & -651.7643 & 0.0022 & 0.0022 & 0.5545 & 0.4074 &
0.0600 & 0.0490 & 2.0212 \\
\end{longtable}

\hypertarget{forecasting}{%
\subsection{Forecasting}\label{forecasting}}

Based on the evaluation of forecasting accuracy and model stability, we
now proceed generate long-term forecasts using the selected model:
\(ARIMA(2,1,0)(2,1,0)_{12}\).

Specifically, we are going to forecast the number of tourism arrivals in
Spain for th 12 months following the last available observation, which
corresponds to November 2018.

So, the forecast covers the period from December 2018 to November 2019
and it includes 95\% confidence intervals to quantify the uncertainty
associated with the predictions.

As it was expected, the months with the highest number of tourist
arrivals correspond to summer season, while the lowest numbers occur
during the winter months. (for more specific info: Fig 26)

\begin{center}\includegraphics[width=1\linewidth]{Tourism-in-Spain_files/figure-latex/unnamed-chunk-24-1} \end{center}

\hypertarget{calendar-effects}{%
\subsection{Calendar Effects}\label{calendar-effects}}

Tourism activity can be influenced by various calendar-related factors,
such as public holidays, school vacations or even the number of weekends
in a month.

To take into consideration these potential influences, we propose
modified models that explicitly incorporate calendar effects into the
analysis by including them as external regressors (Fig 27).

So as to assess the relevance of these variables, we do a t-test on the
corresponding coefficients (Fig 28) where the results show that both
coefficients for the calendar effects are statistically significant.

Additionally, we detect that adding these effects influences the
original model parameters: the coefficient for sar2 is no longer
significant.

Since the model that considers both calendar effects has the lowest AIC
(Fig 27), we choose this one and remove the sar2 coefficient.

So, the final model we adopt is \(ARIMA(2, 1, 0)(1, 1, 0)_{12}\) (Fig
29).

With this final model, we clean our original series of the Easter effect
and traditional calendar effects and, doing this, we obtain an adjusted
series that will undergo the same regular and seasonal differencing.
(the comparison between the original and the adjusted series is shown in
Fig 30).

\hypertarget{outlier-treatment}{%
\subsection{Outlier Treatment}\label{outlier-treatment}}

\hypertarget{automatic-outlier-detection}{%
\subsubsection{Automatic outlier
detection}\label{automatic-outlier-detection}}

To conclude the analysis, we apply an automatic outlier detection
procedure to the series already adjusted for calendar effects. This
allows us to identify and interpret anomalies that may still be present
in the data, thereby refining the model's accuracy.

Some outliers have been identified and the output shows several
informative columns (Fig 31):

\begin{itemize}
\item
  Obs: index of the observation where the outlier is detected.
\item
  Type: kind of outlier detected (AO, LS, TC)
\item
  Weight coefficient: measures the magnitude and direction of the
  anomaly's effect, the larger, the stronger influence.
\item
  Absolute likelihood ratio: significance of the outlier.
\item
  Date: moment when the outlier is detected.
\item
  PercVar: quantifies the deviation of the outlier from the expected
  series.
\end{itemize}

First, in March 2002 a positive additive outlier is detected. That year
the Easter holidays occurred in March, which likely led to an unusual
spike in tourism activity, despite the prior removal of average Easter
effects.

Another positive additive outlier appears in August 2002, suggesting an
exceptionally high volume of arrivals during the summer holiday period,
probably due to favorable conditions or events that boosted travel
beyond typical seasonal patterns. In February 2004, there is a positive
additive outlier too. This can be linked to the fact that 2004 was a
leap year, the additional day in February may have contributed increased
tourist arrivals during this month.

Moving to April 2007, there is a negative transitory change observed.
This may be related to early signs of the global financial crisis, which
began affecting consumer behavior and, consequently, the tourism sector
at that point.

In April 2010, a negative additive outlier appears. This likely reflects
the effects of the volcanic ash cloud in Iceland, which led to the
disruption of air traffic across Europe for several days in April. This
caused a sharp decline in tourism flows during that month.

Furthermore, in April 2012 another negative additive outlier is
detected, likely due to the increase in airport taxes introduced that
month, which may discouraged travel and reduced inbound tourism.

Finally there is a positive level shift identified in April 2013,
suggesting a sustained increase in the level of tourist arrivals from
that point onward. This can be attributed to structural changes in the
tourism sector, marketing strategies, or improvements in economic
conditions that made tourism in Spain more attractive.

After analyzing the detected outliers, we now propose the linearized
series, that is the version of the adjusted series for calendar effects
now also adjusted for outliers.

In the Fig 32 plot, the red line shows the series we would expect if
outliers had not occurred in the adjusted series for calendar effects
and it represents the theoretical behavior of the data in the absence of
anomalous events.

Fig 33 displays the difference between the log-transformed adjusted
series for calendar effects and the linearized version for outliers and
calendar effects. It shows the impact of outliers over time, showing
where and how much the series has been adjusted. The peaks indicate
points where significant outliers were detected and corrected.

Fig 34 shows the differenced linearized series (first a seasonal and
then a regular difference), which have been done in order to achieve
stationarity.

Since the series has changed, we now need to re-examine the ACF and the
PACF plots to determine if we need to reformulate the model.

For the \textbf{seasonal part}, we focus on the lags highlighted in red.

Although it is still challenging to detect a clear and decreasing
pattern, the ACF plot shows unusual behavior at seasonal lags. While the
PACF does not show any significant seasonal spikes, the behavior in the
ACF suggests a potential seasonal AR(1), leading us to consider P = 1.

Turning to the \textbf{regular part}, the lags appear to cut off after
lag 2 in the PACF plot, indicating an AR(2).

Based on what we have said, we can confirm that the model we were
working with is good for the adjusted series.

\begin{center}\includegraphics[width=1\linewidth]{Tourism-in-Spain_files/figure-latex/unnamed-chunk-29-1} \end{center}

To validate our new adjustments and ensure the model is appropriate for
the linearized series, we repeat the estimation of the model and the
residual analysis.

Since our series d1d12lnserieEC.lin has already been differenced, with
both seasonal and regular differencing, we set d = 0 and D = 0 in the
model specification for the estimation.

Like before, we first specify the transformed stationary series (Fig 35)
in order to estimate the mean and check whether it is significant or
not. Since the mean is not significant, we re-estimate the model without
the corresponding differentiations. Instead, we need to specify them in
the estimation method (Fig 36). The model will be called mod3EaTD.lin.

Afterwards, we proceed once again to verify the significance of
coefficients carrying out the t-test, where we observe that all of them
are significant (Fig 37).

In order to validate the model, as we have said before, we repeat the
residual analysis where we have to check:

\begin{itemize}
\item
  \textbf{Homocedasticity (constant variance)}
\item
  \textbf{Normality}
\item
  \textbf{Independence}
\end{itemize}

To verify if \textbf{variance} is constant over time, we plot both the
residuals and the square root of their absolute values, along with a
smoothed trend line. This helps us to observe potential changes in the
spread of the residuals.

We can see that now the red trend line is slightly flatter compared to
the one obtained in the initial analysis. This means that the variance
of the residuals is more stable with the currently adjusted series,
suggesting improved homoscedasticity.

\begin{center}\includegraphics[width=1\linewidth]{Tourism-in-Spain_files/figure-latex/unnamed-chunk-31-1} \end{center}

To assess the \textbf{normality} of the residuals, we proceed with the
Q-Q plot and histogram analysis.

In this case, normality has also improved. The Q-Q plot shows minimal
minimal volatility, and the isolated point observed in the previous
analysis is no longer present, indicating that the outlier has been
removed and, consequently, we obtain a better fit to the normal
distribution.

Concerning the histogram, the tails are now more aligned with the normal
curve. Although the center still displays slightly higher bars than
expected, the overall shape is much closer to a normal distribution.
Therefore, we can conclude that the assumption of normality is
satisfied.

The Shapiro-Wilk normality test confirms our visual analysis, since the
p-value is 0.2349, higher than 0.05.

\begin{center}\includegraphics[width=1\linewidth]{Tourism-in-Spain_files/figure-latex/unnamed-chunk-32-1} \end{center}

Finally, to assess whether the model adequately captures the
autocorrelation structure of the data, we examine the independence of
residuals.

First, we inspect the ACF and PACF plots of the residuals. The
significant spikes that were previously observed in the first analysis
have now been reduced and almost all of them lie within the confidence
bounds, suggesting a better fit.

To complement our assessment of residual independence, we conduct a
Ljung-Box (Fig 38). Unlike the earlier analysis, where p-values from lag
48 onwards remained consistently below the significance threshold, this
time all p-values above 0.05.

This indicates that there is no significant autocorrelation in the
residuals, and thus the model captures the autocorrelation structure
well.

We can now firmly say that residuals are independent.

\begin{center}\includegraphics[width=1\linewidth]{Tourism-in-Spain_files/figure-latex/unnamed-chunk-33-1} \end{center}

After conducting the residual analysis, we can confidently conclude that
now residuals satisfy the assumptions of \textbf{constant variance,
normality and independence}, behaving like white noise.

In order to assess stationarity of the model, we examine the roots of
the characteristic polynomial.

Since invertibility is not applicable (there are no moving average
terms), we are going to focus on the autoregressive components.

Since all roots lie outside the unit circle (Fig 39), we can conclude
that the model is stable and well-specified.

As previously explained, the plot displays the inverse of the roots.
That's the reason why, as expected, the dots lay inside the unit circle,
which confirms that the actual roots lie outside.

\begin{center}\includegraphics[width=1\linewidth]{Tourism-in-Spain_files/figure-latex/unnamed-chunk-34-1} \end{center}

Now that the series has been adjusted for outliers, we generate
forecasts for the original series using the transformed model, which has
gone through both calendar effects and outlier correction.

\hypertarget{forecasts-for-the-original-series}{%
\subsubsection{Forecasts for the original
series}\label{forecasts-for-the-original-series}}

To assess forecasting accuracy, we reserve the last 12 observations for
validation. The model is estimated using the adjusted series, the one
corrected for calendar effects and outliers, while forecasts are
compared against the original (unadjusted) series.

Since we have all the information of the initial model (fitted on the
unadjusted series), we can now compute the same performance metrics for
the transformed model and compare them.

The AIC for the full model is significantly lower when using the cleaned
series, indicating a better overall fit. This improvement is consistent
when comparing both the full and the cut versions of the models, which
confirms that th model is stable.

Similarly, the sigma squared, also called residual variance, is lower in
both versions of the model fitted on the linearized series, suggesting a
reduced unexplained variability.

The values for error metrics are also considerably lower for the model
fitted on the cleaned series, which supports the conclusion that it does
more accurate forecasts. Additionally, the mean confidence interval
width is narrower in the updated model, implying greater precision and
reliability in the forecasts.

\begin{longtable}[]{@{}
  >{\raggedright\arraybackslash}p{(\columnwidth - 18\tabcolsep) * \real{0.1566}}
  >{\raggedleft\arraybackslash}p{(\columnwidth - 18\tabcolsep) * \real{0.1205}}
  >{\raggedleft\arraybackslash}p{(\columnwidth - 18\tabcolsep) * \real{0.1205}}
  >{\raggedleft\arraybackslash}p{(\columnwidth - 18\tabcolsep) * \real{0.0843}}
  >{\raggedleft\arraybackslash}p{(\columnwidth - 18\tabcolsep) * \real{0.0843}}
  >{\raggedleft\arraybackslash}p{(\columnwidth - 18\tabcolsep) * \real{0.0843}}
  >{\raggedleft\arraybackslash}p{(\columnwidth - 18\tabcolsep) * \real{0.0843}}
  >{\raggedleft\arraybackslash}p{(\columnwidth - 18\tabcolsep) * \real{0.0843}}
  >{\raggedleft\arraybackslash}p{(\columnwidth - 18\tabcolsep) * \real{0.0843}}
  >{\raggedleft\arraybackslash}p{(\columnwidth - 18\tabcolsep) * \real{0.0964}}@{}}
\toprule\noalign{}
\begin{minipage}[b]{\linewidth}\raggedright
Model
\end{minipage} & \begin{minipage}[b]{\linewidth}\raggedleft
AIC\_full
\end{minipage} & \begin{minipage}[b]{\linewidth}\raggedleft
AIC\_cut
\end{minipage} & \begin{minipage}[b]{\linewidth}\raggedleft
sig2\_f
\end{minipage} & \begin{minipage}[b]{\linewidth}\raggedleft
sig2\_c
\end{minipage} & \begin{minipage}[b]{\linewidth}\raggedleft
RMSE
\end{minipage} & \begin{minipage}[b]{\linewidth}\raggedleft
MAE
\end{minipage} & \begin{minipage}[b]{\linewidth}\raggedleft
RMSPE
\end{minipage} & \begin{minipage}[b]{\linewidth}\raggedleft
MAPE
\end{minipage} & \begin{minipage}[b]{\linewidth}\raggedleft
CIWidth
\end{minipage} \\
\midrule\noalign{}
\endhead
\bottomrule\noalign{}
\endlastfoot
mod1 & -692.7635 & -651.8284 & 0.0022 & 0.0022 & 0.5535 & 0.4098 &
0.0602 & 0.0497 & 2.0162 \\
mod3EaTD.lin & -844.1085 & -798.0691 & 0.0011 & 0.0011 & 0.2877 & 0.2688
& 0.0480 & 0.0428 & 1.4547 \\
\end{longtable}

Finally, we compare the forecasts from both models for the future. Since
we don't have the actual future data to check forecasting accuracy
numerically, we perform a visual comparison of the predictions.

On the left plot there are the forecasts from the final model,
\(ARIMA(2, 1, 0)(1, 1, 0)_{12}\), fitted to to the linearized series,
which has been adjusted for both outliers and calendar effects.

On the right, there are the forecasts from the previously selected
model, \(ARIMA(2, 1, 0)(2, 1, 0)_{12}\), fitted to the original,
unadjusted series.

The forecasts generated by the adjusted model (left) show smooth and
regular seasonal pattern, closely aligned with the historical behavior
of the series. There are no abrupt spikes or dips and prediction
intervals colored in blue seem reasonably tight.

In contrast, the forecasts from the model fitted on the original series
show a larger amplitude, with higher peaks and variability. This can be
likely caused due to the model being influenced by outliers and calendar
effects in the original data. Forecast confidence intervals seem to be
slightly wider, indicating greater uncertainty.

\begin{center}\includegraphics[width=1\linewidth]{Tourism-in-Spain_files/figure-latex/unnamed-chunk-36-1} \end{center}
\newpage

\hypertarget{conclusions-and-discussion}{%
\section{Conclusions and discussion}\label{conclusions-and-discussion}}

After completing the required analysis for model selection and
validation, we conclude that the model that adjusts best our data is the
\textbf{\(ARIMA(2, 1, 0)(1, 1, 0)_{12}\)}, which has been applied to a
stationary series that was adjusted for both calendar effects and
outliers, ensuring that the input data was clean and representative of
the pattern.

The residuals of the selected model show good properties, resembling
white noise, indicating that the model has successfully captured the
structure of the series.

The model demonstrated good predictive performance, suggesting it
generalizes well beyond the training data.

Through this process, we have developed a model that performs well not
only on historical data but also in forecasting future values, which
reinforces the reliability of our approach.

Apart from the statistical analysis, we also gained insight into various
external factors that can significantly influence series, such as
economic shifts, political events, holidays or even environmental
conditions.

We can conclude that this series is highly sensitive to calendar effects
and outliers, which must be properly addressed so as to produce accurate
forecasts. Moreover, the series presents a clear seasonal pattern, with
a pronounced increase in tourist arrivals during summer months, followed
by a notable decline in colder seasons.

All of these conclusions have been made possible thanks to rigorous
analysis, model selection and cleaning of the data. The ARIMA models
that we have been able to propose, and particularly the final one, have
enabled us to model the true structure of the series effectively.
\newpage

\hypertarget{references}{%
\section{References}\label{references}}

Ministerio de Industria, Comercio y Turismo.
\url{https://sedeaplicaciones.minetur.gob.es/Badase/BadasiUI/lstSeriesInformesPostBack.aspx}
U64A2: FRONTUR ENTRADA DE TURISTAS

RPUbs - Modelos ARIMA. (s.f.). \url{https://www.rpubs.com/Meca/386432}

1.3.5.17. Detection of outliers. (n.d.).
\url{https://www.itl.nist.gov/div898/handbook/eda/section3/eda35h.htm}

Historical events by year - on this day. (n.d.). On This Day.
\url{https://www.onthisday.com/events-by-year.php}

España - Turismo internacional 2002. (n.d.). Datosmacro.com.
\url{https://datosmacro.expansion.com/comercio/turismo-internacional/espana?anio=2002}

Wikipedia contributors. (2025, April 15). 2010 eruptions of
Eyjafjallajökull. Wikipedia.
\url{https://en.wikipedia.org/wiki/2010_eruptions_of_Eyjafjallaj\%C3\%B6kull}

\hypertarget{annex}{%
\section{Annex}\label{annex}}

In the file \emph{TourismSpainScript.Rmd}, you will find the code used
to conduct the analyses and studies related to the project.

Additionally, a PDF document titled \emph{TourismSpainAnnex} contains
all the figures that have been referenced throughout the report.

\end{document}
